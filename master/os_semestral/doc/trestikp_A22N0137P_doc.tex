\documentclass[12pt, letterpaper]{article}
\usepackage[utf8]{inputenc}
\usepackage[czech]{babel}
\usepackage{indentfirst}
\usepackage{listings}
\usepackage{caption}
\usepackage{float}
\usepackage{graphicx}
\usepackage{hyperref}
%%%
%%%
\begin{document}
%%%
%%% TITLE PAGE
%%%
\begin{titlepage}
\centerline{\includegraphics[width=10cm]{img/logo.jpg}}
\begin{center}
\vspace{30px}
{\huge
\textbf{Operační systémy}\\
\textbf{Single task výpočet}\\
\vspace{1cm}
}
{\large
\textbf{KIV/OS}\\
\vspace{1cm}
}
\vspace{1cm}
{\large
Pavel Třeštík\\
}
{\normalsize
A22N0137P
}
\end{center}
\vspace{\fill}
\hfill
\begin{minipage}[t]{7cm}
\flushright
\today
\end{minipage}
\end{titlepage}
%%%
%%% TEXT START
%%%
\section{Úvod}
Cílem práce je implementovat velmi jednoduchý operační systém (dále jen OS), na kterém poběží jeden uživatelský proces
(dále jen task). OS má být cílen na platformu Raspberry Pi Zero. Task komunikuje pomocí rozhraní UART a jeho úkolem je
predikovat hladinu glukózy v intersticiální tekutině. OS musí tasku zajistit přístup k UART, poskytnout paměť k
provádění predikce a zajistit, že task nemůže zasahovat do paměti, která mu nepatří. Pokud task zrovna neprovádí
výpočet, tak se procesor přepne do úsporného režimu.
%
\section{Analýza}
Pro predikci task využívá genetického algoritmu. Genetický algoritmus je ale téměř nemožné implementovat bez použití
generátoru náhodných čísel. Ten je v algoritmu použit alespoň pro mutaci prvků, ale také je vhodné ho použít pro
náhodnou inicializaci chromozomů (prvků algoritmu).

Pro task tedy musí být minimálně možné pracovat s UARTem, generovat náhodná čísla a přidělovat paměť, kterou si task 
alokuje při své inicializaci.

Raspberry Pi Zero (dále jen RPi) je osazeno System on a 
Chip\footnote{\url{https://en.wikipedia.org/wiki/System_on_a_chip}} Broadcom BCM2835, který poskytuje 
práci s periferiemi, kterými je RPi osazeno. Mezi tyto periferie patří např. Serial Peripheral Interface (SPI), Pulse
Width Modulation (PWM), časovače a další.

RPi nabízí plnohodnotný UART, ale i verzi miniUART skrze Auxilární koprocesor (AUX). UART je poskytnut na General 
Purpose Input Output (GPIO) rozhraní a proto je třeba implementovat ovladač pro alespoň tyto GPIO, UART a AUX.

Generátor náhodných čísel je možné implementovat pomocí PWM.

Pro ochranu zařízení, tedy aby uživatel neměl neomezenou kontrolu, CPU běžně poskytují různé režimy. RPi poskytuje
7 režimů, z nichž 6 je privilegovaný (např. System (SYS), IRQ, FIQ, SVC...) a 1 je neprivilegovaný (User (USR)).
Tato práce má zajistit alespoň základní ochranu paměti a celkově zařízení, takže potřebujeme umět přepínat mezi USR režimem a 
privilegovanými režimy. V *nixových systémech je procesor v USR režimu a pokud potřebuje přístup do privilegovaného
režimu, tak je typicky volána knihovní funkce, která vyvolá přerušení, čímž přepne procesor do privilegovaného
režimu. U *nixů je se vším navíc zacházeno jako se souborem, takže základní ovládání všech zdrojů je dáno jednoduchým
rozhraním s funkcemi Open, Close, Read, Write. Pro jednoduchost implementace je tuto vhodný způsob jak v této práci
řešit používání periferií a přepínání mezi privilegovanými a neprivilegovaným režimem.
%
\section{Implementace}
Práce je implementována nad poskytnutou kostrou KIV-RTOS\footnote{\url{https://github.com/MartinUbl/KIV-RTOS}}. 
V této kostře jsou již připraveny 
některé ovladače, souborový systém, přepínání procesů, plánovač a příkladové uživatelské procesy. Z připravených
věcí tato práce ale nevyužije zdaleka všechny. 
%
\subsection{Kernel}
Kostra již obsahuje ovladač pro UART, GPIO, AUX, filesystem (FS), procesy, generování náhodných čísel a obsluhy
přerušení. Některé z těchto ovladačů bylo třeba rozšířit.
\subsubsection{UART}
Protože je využit miniUART, tak potřebujeme AUX ovladač a GPIO. Tyto dva ovladače už naštěstí není třeba rozšiřovat.
Samotný UART ovladač ale bylo třeba rozšířit. Do ovladače byla dodělána funkce čtení UART zařízení pomocí přerušení,
které je potřeba pro volání skrz FS.

Při vyvolání přerušení je obsah FIFO registru, který je poskytnut, přepsán do kruhového bufferu. Kruhový buffer je
implementován tak, že do pole fixní délky zapisuje a při dosažení konce přetéká zpět na začátek. V bufferu jsou 
dva indexy - psací a zapisovací. Původní implementací bylo, že psací index nesmí \uv{předehnat} čtecí o celou délku 
pole, protože to by přepsalo dosud nepřečtená data. To ale bylo změněno a v odevzdané verzi je toto umožněno. Důvodem
je, že pokud načítáme vstup po řádce, tak při zadání vstupu, který je delší než délka bufferu, by do pole nikdy
nebyl vložen konec řádku a tím by nikdy nedošlo ke přečtení a \uv{vyprázdnění} bufferu.

Přijaté znaky jsou také vypsány zpět na konzoli, aby uživatel viděl, co za vstup zadal.
%
\subsubsection{Procesy a paměť}
Při vytváření procesu se alokuje tabulka stránek a do této tabulky byly zavedeny dvě stránky pro 
kód procesu a jeho stack. Memory management unit (MMU) pracuje s touto tabulkou a převádí virtuální adresy na 
fyzické. Toto již bylo implementováno, ale byla potřeba to rozšířit.

Rozšířením je, že do struktury procesu byl předán pointer na tabulku stránek, aby se do ní mohly zavést nové stránky,
které si alokuje uživatelský proces. Pokud se uživatelský proces pokusí zasáhnout do paměti, která mu nepatří, tak
systém ukončí proces a končí v cyklu. Alokace paměti je tedy taková, že uživatelský proces volá \texttt{malloc}, který
vyvolá přerušení s nově zavedeným kódem. Obsluha tohoto přerušení volá \texttt{sbrk}, což je funkce, která vrátí 
pointer na alokovanou paměť. Pokud se vyžádaná paměť vejde do již alokované paměti, tak \texttt{sbrk} nealokuje novou
stránku. Pokud \texttt{sbrk} přiřazuje nové stránky, tak alokuje po jedné, protože používáme 1MB stránky, což je
poměrně velký blok. Pokud je vyžádaná paměť větší než paměť stránky samotné, tak \texttt{sbrk} alokuje
stránky, dokud nemá dostatek paměti.

V kostře je implementován jednoduchý plánovač procesů. Ačkoliv má na RPi běže pouze jediný už. task, tak je vhodné
plánovač zachovat, protože plánovač přepíná procesor do úsporného režimu, pokud nemá co plánovat. Toto je přesně 
chování, které se pro tuto práci hodí. Implementaci plánovače nebylo třeba rozšiřovat. K využití přepnutí do 
úsporného režimu stačí zablokovat běžící task. Pokud probíhá výpočet, tak je UART rozhraní čteno bez 
blokování - tedy pokud na zařízení UART nic není, tak proces pokračuje ve své činnosti. Pokud ale task zrovna 
nepočítá, tak se UART čte blokujícím způsobem. Toho je v práci dosaženo voláním \texttt{wait()} nad UART rozhraním. 
\texttt{wait()} zařídí uspání procesu, který ho volá do doby, dokud z žádaného zařízení nepřijde interrupt, který
skrz FS volá \texttt{notify()}.
%
\subsection{Userspace task}
Na začátku procesu je vypsána uvítací hláška, která sděluje informace o práci a instrukce k použití. Následně jsou
načteny 2 celočíselné hodnoty (\texttt{t\_delta} a \texttt{t\_pred}) a po jejich načtení je alokována paměť 
potřebná k provádění predikcí užitím genetického
algoritmu (GEA). Poté začne hlavní smyčka programu, z které by task už nikdy neměl vyskočit. Již bylo zmíněno, že 
se predikuje hladina glukózy v intersticiální tekutině. Tato predikce je počítána modelem 
$y(t + t\_pred) = A * b(t) + B * b(t) * (b(t) - y(t)) + C$, kde $b(t)$ se počítá jako: 
$b(t) = D/E * dy(t)/dt + 1/E * y(t)$. Derivaci $dy(t)/dt$ v práci stačí aproximovat diferencí ($t\_delta$) 
dělenou číslem $1.0 / (24.0 * 60.0 * 60,0)$, je-li $t$ v sekundách.

Hlavní smyčka programu vypadá zhruba následovně. Nejdříve načítá vstup pomocí blokujícího čtení. Pokud je přijatý vstup 
číslo, tak začne výpočet. Výpočet provádí 100 generací, které výše popsanou rovnicí počítají fitness funkci 
jednotlivých chromozomů generace a tím ladí parametry \texttt{A-E}. Po ohodnocení generace se vytvoří nová
generace ze stávající. Nejlepších 10\% chromozomů je zachováno. Dalších přibližně 80\% generace je zaplněno křížením
chromozomů. Zbytek nové generace je doplněn nejlepšími chromozomy. 40\% odzadu nové generace je mutováno, proto 
je doplnění generace prováděno nejlepšími chromozomy. Po ohodnocení generace a vytvoření nové generace se vracíme
do hlavního cyklu, kde je tentokrát bez blokování přečteno rozhraní UART. Pokud je přijata řádka, je vstup parsován
a na základě vstupu je provedena akce. Pokud přijde vstup \uv{stop} - tedy příkaz k přerušení výpočtu, tak se ukončí 
probíhající výpočet a vypočte se predikce použitím poslední generace výpočtu, který doběhl kompletně. Pokud přijde 
příkaz \uv{stop} kdykoliv, kdy se nepočítá, tak se pouze vypisuje chybové hlášení. Pokud na vstupu je číslo a výpočet již
probíhá tak se také vypisuje chybové hlášení. Pokud je na vstupu řetězec \uv{parameters} tak se vypíší parametry momentálně
počítané generace. Tento příkaz funguje kdykoliv v hlavním cyklu a teoreticky je i možné vypisovat parametry, jak se
mění během výpočtu. Pokud na vstupu nic není, tak cyklus pokračuje počítáním další ze 100 generací, nebo pokud 
byla dopočítána 100tá generace, tak se přejde do stavu blokujícího čekání na vstup.
%
\section{Řešené problémy}
V práci bylo řešenou pouze několik menších problémů. Jeden takovýto problém byl již popsán v implementaci UART a tím
je čtení řádky z kruhového bufferu, pokud bychom bufferu neumožňovaly přetéct o celou délku pole. Řešením bylo
psát do bufferu, dokud není zapsán celý vstup bez ohledu na délku zapsaných dat. Pokud by buffer přetekl a nebo byla
zahozena část zprávy, která by se do bufferu nevešla, už je celkem nepodstatné, protože v obou případech je vstup
neúplný.

Druhým problémem byly statické proměnné ve třídě \texttt{Chromosome}. Tyto proměnné v tasku nefungovali a při debugování
bylo zjištěno, že program končí ve smyčce chyby práce s pamětí. Není známa příčina, ale tento problém byl vyřešen
předěláním statických proměnných na globální proměnné, pro které byl udělán namespace \texttt{Chromosome\_NS}.

Třetím problémem bylo otevření generátoru náhodných čísel, který také končil ve smyčce chyby práce s pamětí. Důvodem
bylo, že při inicializaci (otevření) generátoru se povolovalo přerušení pro něj, ale alespoň v qemu bylo zjištěno, že
adresa povolení přerušení je nepřístupná. U tohoto problému nebylo vyřešeno proč je adresa nepřístupná, ale pro tuto
práci generátor funguje i bez přerušení. Přerušení generátoru se nepovolují.
%
\section{Testování}
Práce byla vyvíjena na emulátoru \texttt{qemu} a debugována připojením debuggeru \texttt{gdb} na emulátor. 
Task byl testován pouze manuálně, ale díky jeho jednoduchosti by toto testování mělo být dostatečné. Funkčnost
kernelu ve své podstatě testuje task jako takový, protože z tasku se volají kernelové funkce, které když nefungovali,
tak byly debugovány a opraveny.

Byl i neúspěšný pokus práci zprovoznit na hardware. Problémem se zprovo\-zněním na hardware je to, že se nepovedlo
navázat UART komunikaci. Bylo ujištěno, že klient používá stejné nastavení, tedy baud rate 115200 a žádný paritní bit.
Kvůli nedostatku času nebylo zprovoznění na hardware prioritou.
%
\section{Závěr}
Práce staví na kostře KIV-RTOS, kde již byla předpřipravená velká část kernelu a také zde byly příkladové userspace 
tasky. Do kernelu bylo do-implementováno čtení UART rozhraní a alokace paměti pro uživatelský proces. Následně byly 
vytvořeny obdoby standardních funkcí, které byly potřeba pro task. Mezi tyto funkce patří například \texttt{ftoa},
\texttt{fgets}, ale i třída \texttt{Random} pro generování náhodných čísel.

Poté byl naimplementován samotný task, který použitím genetického algoritmu ladí parametry predikčního modelu.
Práce by měla splňovat všechny požadované body zadání.
%%%
%%%
\end{document}
