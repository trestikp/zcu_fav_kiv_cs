\documentclass[12pt, letterpaper]{article}
\usepackage[utf8]{inputenc}
\usepackage[czech]{babel}
\usepackage[normalem]{ulem}
\usepackage{amsmath}
\usepackage{indentfirst}
\usepackage{listings}
\usepackage{caption}
\usepackage{float}
\usepackage{graphicx}
\usepackage{hyperref}
\usepackage{textcomp}
\usepackage{array}
%%%
%%%
\begin{document}
%%%
%%% TITLE PAGE
%%%
\begin{titlepage}
\centerline{\includegraphics[width=10cm]{img/logo}}
\begin{center}
\vspace{30px}
{\huge
\textbf{Programování Internetových aplikací}\\
\vspace{1cm}
}
{\Large
\textbf{KIV/PIA-E}\\
\vspace{1cm}
}
{\large
\textbf{Bikesharing platform}\\
\vspace{1cm}
}
\vspace{1cm}
{\large
Pavel Třeštík\\
}
{\normalsize
A23N0001P
}
\end{center}
\vspace{\fill}
\hfill
\begin{minipage}[t]{7cm}
\flushright
\today
\end{minipage}
\end{titlepage}
%%%
%%% TEXT START
%%%
\section{Realizace projektu}
Projekt se skládá z modulů backendu a frontendu. Navíc jsou použity služby ActiveMQ pro komunikaci klientů a serveru
a MySQL databáze.
\subsection{Backend}
Backend je jeden samostatný modul. Jedná se tedy o monolit, což bylo záměrně zvoleno s ohledem na to, že projekt není
příliš rozsáhlý. Struktura modulu je pak dělena na 3 vrstvy: data, business (logic), presentation (communication).
Ačkoliv je modul monolit, tak použitá 3 vrstvová architektura stále zaručuje rozumnou strukturu projektu.

Produkt je implementován použitím Java17, Spring Boot 3 a Flyway. K backendu je také potřeba databáze, která využívá
MySQL docker image. Celý projekt je pak sestavěn nástrojem Maven. Navíc je také použito několik knihoven třetích stran.
%
\subsubsection{Konfigurace}
Backend umožňuje konfiguraci důležitých property použitím konfiguračních souborů, které Spring načítá.
Jedná se o soubory \texttt{bikesharing-backend/src/
main/resources/application.properties} a 
\texttt{bikesharing-backend/src/
main/resources/application.yaml}. Jednotlivé property, které je možné konfigurovat jsou
okomentované v souborech, případně jejich jméno by mělo jasně vypovídat o jejich účelu.

Flyway také používá skripty z adresáře \texttt{bikesharing-backend/src/main/
resources/db/migration}. Zde jsou 2 skripty,
první inicializuje strukturu databáze a druhý inicializuje základní data, která mohou být použita k testování aplikace.
%
\subsection{Frontend}
Frontend je také samostatný modul. Protože frontend je implementovaný využitím Angular 17, tak je dodržena struktura 
projektu, kterou vygeneruje Angular příkaz pro založení projektu. Tato struktura odděluje komponenty 
(view vrsta z šablony, stylů a příslušné obsluhy), service a model.

Kromě čístého Angular 17 jsou použity knihovny rozšiřujích Angular například o tabulky. Jsou použity i knihovny 
třetích stran, například Leaflet pro zobrazovnání map a markerů.
%
\subsubsection{Konfigurace}
Frontend také má konfigurační soubor, pro umožnění dynamického nastavení. Například aby se klient mohl připojit na jiný
backend server. Jedná se o soubor \texttt{bikesharing-ui/src/environments/environment.prod.ts}.
%
\subsection{Další technologie}
Pro databázi je využito MySQL 8. Pro MQ je využito Apache ActiveMQ. Pro build a nasazení práce je použit Docker, resp.
docker-compose. Respektive alternativa dockeru Podman, ovšem tyto nástroje jsou téměř zaměnitelné, takže funkčnost by
měla být stejná.

Každý modul práce a technologie má vlastní kontajner, takže výsldek jsou tedy 4 kontajnery: bikesharing-backend, 
bikesharing-ui, mysql a activemq.
%%%
%%% 
%%%
\section{Spuštění}
Tento proces je také popsán v README, trochu stručněji.

Build a spuštění by měl být jedním příkazem: \textbf{podman-compose up --build} (resp. docker). Tento příkaz vybuildí
image pro backend a frontend projektu. Poté spustí tyto image navíc s MySQL a ActiveMQ imagi. Spuštění všech kontejnerů
do naběhnutého stavu může zabrat několik desítek vteřit. Build je v rámci jednotek minut.

Po spuštění by pak měla být aplikace dostupná na adrese \url{http://localhost:4200}, za předpokladu, že je použita
výchozí konfigurace. Je doporučené zachovat tuto adresu, protože GitHub OAuth2 aplikace vyžaduje adresu, na kterou se
má přesměrovat po úspěšné authentikaci. Ačkoliv je možné tuto adresu konfigurovat soubory, tak je adresa vázána na 
GitHub aplikaci a při změně by tato adres přestala odpovídat a GitHub by tím pádem zamítl jakýkoliv request. Je ovšem
možné vytvořit si vlastní GitHub aplikaci pro authentikaci, kde se zadá jiná adresa aplikace a po změně konfigurace 
pak bude fungovat nová adresa.
%
\subsection{Problém buildu frontend image}
Při vývoji se objevil jeden problém s buildem frontend image. Tento problém nastává při kroku \textbf{npm install},
kdy tento krok "běží do nekonečna". Není znám důvod problému a nastává pouze v jedné specifické internetové
síti, takže problém by pravděpodobně neměl nastat. 

Problém je možné obejít tím, že se \textbf{npm install} provede na host stroji. Tedy změnit adresář do 
\texttt{bikesharing-ui} a spustit \texttt{npm install}. Předpoklad je, že na host stroji je dostupný nástroj 
\texttt{npm}.
%%%
%%%
%%%
\section{Použití}
Tento proces je také popsán v README, trochu stručněji. Podobný návod je také na stránce "About" aplikace.

Hlavní stránkou aplikace je mapa. Na této mapě jsou zobrazeny standy pomocí markeru. Světle modré markery s číslem
jsou standy, které mají kola a může z nich započít jízda. Oranžové markery jsou standy bez kol a mohou jízdu pouze
přijmout.

Při zahájení jízdy je na mapě modře znázorněna cesta jízdy a zelený marker s kolem, který simuluje uživatelovo jízdu
po zvýrazněné cestě. Probíhající jízda se zobrazuje všem klientům jako šedý marker s kolem.

Standy a pozorování probíhajících jízd jsou zobrazeny všem uživatelům. Pro správně vyzkoušení je doporučeno
otevřít alespoň 2 okna s klientem a alespoň v jednom spustit jízdu.

Spouštění jízdy je prováděno ze stránky z mapou, kdy po přihlášení se uživately zobrazí 2 select boxy a tlačítko pro
spuštění jízdy.

Pokud je uživatel přihlášen, tak si může zobrazit seznam svých jízd na stránce "Past rides".

Pokud je přihlášený uživatel navíc SERVICEMAN, tak může zobrazit seznam kol, které potřebují servisovat a v tomto
seznamu je může označit jako servisované.

Pro testování jsou poskytnuti 2 uživatelé \textbf{test} a \textbf{test2}, který má navíc roli SERVICEMAN. 
Oba uživatelé mají heslo \textbf{test123}.
%
\end{document}
