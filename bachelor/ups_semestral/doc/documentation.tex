\documentclass[12pt]{report}


% Zařadit literaturu do obsahu
\usepackage[nottoc,notlot,notlof]{tocbibind}

% Umožňuje vkládání obrázků
\usepackage[pdftex]{graphicx}

% Odkazy v PDF jsou aktivní; navíc se automaticky vkládá
% balíček 'url', který umožňuje např. dělení slov
% uvnitř URL
\usepackage[pdftex]{hyperref}
\hypersetup{colorlinks=true,
  unicode=true,
  linkcolor=black,
  citecolor=black,
  urlcolor=black,
  bookmarksopen=true}

% Při používání citačního stylu csplainnatkiv
% (odvozen z csplainnat, http://repo.or.cz/w/csplainnat.git)
% lze snadno modifikovat vzhled citací v textu
\usepackage[numbers,sort&compress]{natbib}


% me package
\usepackage{float}
\usepackage{subfig}
\usepackage{siunitx}
\usepackage[czech]{babel}
\usepackage{tabularx}

%%%%%%%%%%%%%%%%%%%%%%%%%%%%%%%%%%%%%%%%%%%%%%%%%%%%%%%%%%
%
% VLASTNÍ TEXT PRÁCE
%
%%%%%%%%%%%%%%%%%%%%%%%%%%%%%%%%%%%%%%%%%%%%%%%%%%%%%%%%%%

\makeatletter
\def\@makechapterhead#1{%
  {\parindent \z@ \raggedright \normalfont
    \Huge\bfseries
    \ifnum \c@secnumdepth >\m@ne
        \hangindent=1.5em
        \noindent\hbox to 1.5em{\thechapter\hfil}%
    \fi%
    #1\par\nobreak
    \vskip 40\p@
  }}
\def\@makeschapterhead#1{%
  {\parindent \z@ \raggedright \normalfont
    \interlinepenalty\@m
    \huge \bfseries  #1\par\nobreak
    \vskip 40\p@
  }}
\makeatother


\begin{document}
%

\begin{titlepage}
\centerline{\includegraphics[width=10cm]{img/logo.jpg}}
\begin{center}
\vspace{30px}
{\huge
\textbf{úvod do počítačových sítí}\\
\textbf{hra: dáma (anglická verze)}\\
\vspace{1cm}
}
{\large
\textbf{kiv/ups}\\
\vspace{1cm}
}
\vspace{1cm}
{\large
pavel třeštík\\
}
{\normalsize
a17b0380p
}
\end{center}
\vspace{\fill}
\hfill
\begin{minipage}[t]{7cm}
\flushright
\today
\end{minipage}
\end{titlepage}


%\maketitle
\tableofcontents



\chapter{popis hry dáma (anglická verze)}
jako téma semestrální práce jsem si zvolil hru dáma. zvolil jsem takzvanou anglickou dámu. tedy
hra se hraje na šachovnici s 64mi poli a vyhrává ten hráč, který první sebere všechny
kameny protivníka. hráči mají kameny na černých polích šachovnice a smí se pohybovat pouze
diagonálně, tedy pouze po černých polích. kamenů jsou dva druhy: pěšák a král. rozdíl mezi
klasickou verzí dámy a anglickou dámou je, že v anglické verzi král (česky označován jako dáma),
smí všemi směry, ale stejně jako pěšák pouze o jedno políčko. mezitím v klasické verzi
král (dáma) může všemi směry a skočit libovolný počet polí.

\section{pohyb}
pěšák smí pouze o jedno políčko v před. král (dáma) smí o jedno políčko všemi směry. tah hráče
končí po posunutí kamene.

\begin{figure}[H]
  \centering
  \begin{minipage}[b]{0.4\textwidth}
    \includegraphics[width=\textwidth]{img/move_a.png}
    \caption{Zvýraznění tahu}
  \end{minipage}
  \hfill
  \begin{minipage}[b]{0.4\textwidth}
    \includegraphics[width=\textwidth]{img/move_b.png}
    \caption{Po tahu}
  \end{minipage}
\end{figure}

\section{Braní kamenů protivníka}
Pokud v cestě ku předu stojí kámen protivníka a je za ním prázdné políčko, tak jak král, tak pěšák 
přeskočí kámen protivníka na políčko za něj a kámen protivníka odstraní. Stále platí, že
pěšák smí brát pouze ku před, zatímco král může brát všemi směry.

\begin{figure}[H]
  \centering
  \begin{minipage}[b]{0.4\textwidth}
    \includegraphics[width=\textwidth]{img/take_a.png}
    \caption{Sebrání protivníkova kamene}
  \end{minipage}
  \hfill
  \begin{minipage}[b]{0.4\textwidth}
    \includegraphics[width=\textwidth]{img/take_b.png}
    \caption{Po provedení tahu}
  \end{minipage}
\end{figure}

\section{Pohyb}
Pokud po tom, co hráč vzal protivníkův kámen, je v cestě další protivníkův kámen, za kterým
je volné místo, hráč táhne znovu a vezme tím tak 2 protivníkovy kameny v jednom tahu. Toto
je možné opakovat dokud je splněna ta podmínka, že v blízkosti kamene je protivníkův kámen
a za ním je volné políčko. Tedy hráč může vykonat další tah, pokud během dalšího tahu sebere
kámen protivníka.

\begin{figure}[H]
  \centering
  \begin{minipage}[b]{0.4\textwidth}
    \includegraphics[width=\textwidth]{img/chain_take_a.png}
    \caption{Řetězové braní dvou kamenů}
  \end{minipage}
  \hfill
  \begin{minipage}[b]{0.4\textwidth}
    \includegraphics[width=\textwidth]{img/chain_take_b.png}
    \caption{Po provedení tahu}
  \end{minipage}
\end{figure}
%
%
%
\chapter{Popis protokolu}
Protokol je posílán jako nešifrovaný text. Jednotlivé části jsou oddělené znakem '$|$'. Zpráva
je ukončena znakem '\textbackslash n'.
%
%
\section{Obecný tvar zpráv}
Obecný tvar zprávy je:
\begin{itemize}
	\item ID\_hráče$|$INSTRUKCE$|$parametr$|$parametr...
\end{itemize}

\noindent Obecný tvar odpovědi serveru je: 

\begin{itemize}
	\item ID\_hráče$|$VYSLEDEK$|$kód\_výsledku$|$zpráva$|$parametr$|$parametr...
\end{itemize}


\noindent VYSLEDEK nabývá hodnot "OK" a "ERROR", lze ho považovat za typ instrukce.
%
%
\section{Tabulka instrukcí a stavový diagram protokolu}

\begin{table}[H]
\begin{tabularx}{\textwidth}{|c|X|}
\hline
\textbf{Instrukce}      & \textbf{Popis}                                                                                                                                          \\ \hline
\textbf{CONNECT}        & Požadavek na vytvoření hráče. Očekávána jako první instrukce komunikace. Server odpojí kterékoliv připojení, které jako první instrukci nepošle CONNECT \\ \hline
\textbf{LOBBY}          & Požadavek pro získání seznamu hracích místností, které jsou dostupné k připojení.                                                                       \\ \hline
\textbf{CREATE\_LOBBY}  & Požadavek na vytvoření nové herní místnosti.                                                                                                            \\ \hline
\textbf{DELETE\_LOBBY}  & Požadavek o zrušení existující herní místnosti.                                                                                                         \\ \hline
\textbf{JOIN\_GAME}     & Požadavek na připojení k zvolené herní místnosti.                                                                                                       \\ \hline
\textbf{TURN}           & Požadavek na provedení tahu.                                                                                                                            \\ \hline
\textbf{DISCONNECT}     & Požadavek na odpojení spojení.                                                                                                                          \\ \hline
\textbf{PING}           & Ping slouží k udržení stávající komunikace. Server na PING také odpovídá instrukcí PING.                                                                \\ \hline
\textbf{OPPONENT\_JOIN} & Instrukce serveru informující protivníka hráče, který zaslal požadavek o připojení protihráče.                                                          \\ \hline
\textbf{OPPONENT\_DISC} & Instrukce serveru informující protivníka hráče o protivníkově ztrátě připojení.                                                                          \\ \hline
\textbf{OPPONENT\_TURN} & Instrukce serveru informující protivníka hráče o tahu provedeným hráčem.                                                                                 \\ \hline
\textbf{OPPONENT\_RECO} & Instrukce informující protivníka hráče o úspěšné znovu připojení hráče do hry.                                                                          \\ \hline
\textbf{OPPONENT\_LEFT} & Instrukce informující hráče o protivníkově permanentní ztrátě připojení a ukončení probíhající hry.                                                     \\ \hline
\end{tabularx}
	\caption{Tabulka instrukcí a popisů}
	\label{tab:instruction_tab}
\end{table}

\begin{figure}[H]
	\centering
	\includegraphics[width=\textwidth]{img/protocol_auto.png}
	\caption{Diagram komunikace protokolem}
	\label{img:protocol_auto}
\end{figure}

\section{Popis instrukcí a odpovědí}
%
\subsection{Obecné negativní odpovědi serveru}
\begin{itemize}
	\item 400 - Sockets don't match
	\item 400 - Unknown connection
	\item 400 - Verification failed
	\item 401 - Instruction got too many parameters
	\item 401 - Unrecognized instruction
	\item 401 - TURN needs at least 2 parameters
	\item 401 - Too many parameters for TURN
\end{itemize}
%
\subsection{Instrukce: CONNECT}
\subsubsection{Popis}
Připojí hráče k serveru. Pokud první zpráva od nového připojení není CONNECT,
tak server spojení odpojí. Při zadání ID hráče slouží k znovu připojení k serveru.
Při znovu připojování k serveru, server posílá dodatečné informace nutné k obnovení
klienta do stavu před ztrátou připojení.

\subsubsection{Přesné formáty}
\begin{itemize}
	\item 0$|$CONNECT$|$username - hráč se připojuje poprvé
		\begin{itemize}
			\item username - jméno hráče
		\end{itemize}
	\item ID\_hráče$|$CONNECT$|$username - hráč se pokouší znovu připojit
		\begin{itemize}
			\item username - jméno hráče
		\end{itemize}
\end{itemize}

\subsubsection{Kódy a zprávy odpovědí}
\begin{itemize}
	\item Pozitivní
		\begin{itemize}
			\item 201 - Connection success
			\item 202 - Reconnection success$|$state
			\item 202 - Reconnection success$|$state$|$board$|$on\_top$|$opponent\_name
				\begin{itemize}
					\item state - stav hráče na serveru (connected, turn, opponent\_turn)
					\item board - hrací pole (pouze černé pole - délka 32 znaků)
					\item on\_top - jestli má hráč červené/ modré kameny (0 = modré, 1 = červené)
					\item opponent\_name - jméno protivníka
				\end{itemize}
		\end{itemize}
	\item Negativní
		\begin{itemize}
			\item 402 - Username is empty
			\item 403 - Username too long
			\item 404 - Username already in use
			\item 405 - Server failed to add player
			\item 406 - Player with this ID doesn't exist
			\item 407 - Is this an attack attempt
			\item 408 - Failed to attach gameboard
			\item 409 - Maximum number of connections reached
			\item 410 - This socket is already connected
		\end{itemize}
\end{itemize}
%
%
%
\subsection{Instrukce: CREATE\_LOBBY}
\subsubsection{Popis}
Vytvoří nové lobby a přidá hráče, který ho vytvořil, jako hráč 1. Hráč 1 má vždy
bílé (v mé verzi modré) kameny.

\subsubsection{Přesný formát}
\begin{itemize}
	\item ID\_hráče$|$CREATE\_LOBBY$|$lobby\_name
		\begin{itemize}
			\item lobby\_name - jméno vytvářené místnosti
		\end{itemize}
\end{itemize}

\subsubsection{Kódy a zprávy odpovědí}
\begin{itemize}
	\item Pozitivní
		\begin{itemize}
			\item 201 - Successfully created lobby
		\end{itemize}
	\item Negativní
		\begin{itemize}
			\item 402 - Server failed to create lobby
			\item 403 - Lobby name is too long
			\item 404 - Lobby name already exists
			\item 405 - This cannot be done in current state
			\item 406 - Failed to add player to game
			\item 407 - Failed to add game
		\end{itemize}
\end{itemize}
%
%
%
\subsection{Instrukce: JOIN\_GAME}
\subsubsection{Popis}
Připojí hráče do existující místnosti. Připojeného hráče nastaví jako hráč 2, který má
černé (v mé verzi červené) kameny. Místnost, do které se hráč připojil je odstartována.

\subsubsection{Přesný formát}
\begin{itemize}
	\item ID\_hráče$|$JOIN\_GAME$|$lobby\_name
		\begin{itemize}
			\item lobby\_name - jméno místnosti pro připojení
		\end{itemize}
\end{itemize}

\subsubsection{Kódy a zprávy odpovědí}
\begin{itemize}
	\item Pozitivní
		\begin{itemize}
			\item 201 - Successfully joined game
		\end{itemize}
	\item Negativní
		\begin{itemize}
			\item 402 - Lobby name is too long
			\item 403 - This cannot be done in current state
			\item 404 - Failed to find game lobby
			\item 405 - Failed to contact opponent
			\item 406 - Server lost game
			\item 407 - Failed to add game
		\end{itemize}
\end{itemize}
%
%
%
\subsection{Instrukce: DELETE\_LOBBY}
\subsubsection{Popis}
Zruší vytvořené lobby, pokud v něm je pouze zakladatel lobby (teoreticky nemůže
existovat lobby s více hráči než 1, protože hra je automaticky odstartována po
připojení druhého hráče).

\subsubsection{Přesný formát}
\begin{itemize}
	\item ID\_hráče$|$DELETE\_LOBBY
\end{itemize}

\subsubsection{Kódy a zprávy odpovědí}
\begin{itemize}
	\item Pozitivní
		\begin{itemize}
			\item 201 - Lobby deleted
		\end{itemize}
	\item Negativní
		\begin{itemize}
			\item 402 - This cannot be done in current state
			\item 402 - You don't have lobby
			\item 402 - No game found
		\end{itemize}
\end{itemize}
%
%
%
\subsection{Instrukce: LOBBY}
\subsubsection{Popis}
Požadavek uživatele na získání místností dostupných k připojení.

\subsubsection{Přesný formát}
\begin{itemize}
	\item ID\_hráče$|$LOBBY
\end{itemize}

\subsubsection{Kódy a zprávy odpovědí}
\begin{itemize}
	\item Pozitivní
		\begin{itemize}
			\item 201 - Available lobbies$|$lobby\_name\_1$|$...
		\end{itemize}
	\item Negativní
		\begin{itemize}
			\item 402 - Failed to fetch game
			\item 403 - This cannot be done in current state
		\end{itemize}
\end{itemize}
%
%
%
\subsection{Instrukce: TURN}
\subsubsection{Popis}
Provede tah nebo sekvenci tahů. Kontaktuje protivníka o hráčovo tazích.

\subsubsection{Přesný formát}
\begin{itemize}
	\item ID\_hráče$|$TURN$|$parametr\_1$|$parametr\_2$|$...$|$parametr\_30
		\begin{itemize}
			\item parametr\_1 - index (pozice) kamene
			\item parametr\_x - index (pozice) cíle tahu
					  - index nabývá hodnot 0-63
		\end{itemize}
\end{itemize}

\subsubsection{Kódy a zprávy odpovědí}
\begin{itemize}
	\item Pozitivní
		\begin{itemize}
			\item 202 - Turn Successful
			\item 203 - You won!
			\item 204 - You lost!
		\end{itemize}
	\item Negativní
		\begin{itemize}
			\item 402 - This cannot be done in current state
			\item 403 - Failed to find game
			\item 404 - It is not your turn
			\item 405 - Need starting position
			\item 406 - Too few parameters
			\item 407 - Parameter isn't number
			\item 408 - Failed to validate move
			\item 409 - Failed to find opponent
			\item 410 - Failed to contact opponent
			\item 411 - Opponent message error
		\end{itemize}
\end{itemize}
%
%
%
\subsection{Instrukce: DISCONNECT}
\subsubsection{Popis}
Použito když se hráč odpojuje od serveru. Vymaže hráče ze seznamu hráčů. Zavře spojení z kterého 
přišel požadavek.

\subsubsection{Přesný formát}
\begin{itemize}
	\item ID\_hráče$|$DISCONNECT
\end{itemize}

\subsubsection{Kódy a zprávy odpovědí}
\begin{itemize}
	\item Pozitivní
		\begin{itemize}
			\item 201 - You were disconnected
		\end{itemize}
	\item Negativní
		\begin{itemize}
			\item nemá negativní odpovědi. Vždy zavře připojení, z kterého přišel požadavek.
		\end{itemize}
\end{itemize}
%
%
%
\subsection{Instrukce: OPPONENT\_JOIN}
\subsubsection{Popis}
Instrukce, kterou posílá server klientovi (protivníkovi hráče, který poslal instrukci JOIN\_GAME),
informující ho o připojení protivníka.

\subsubsection{Přesný formát}
\begin{itemize}
	\item ID\_protivníka$|$OPPONENT\_JOIN$|$kód$|$zpráva$|$username
		\begin{itemize}
			\item kód - jedná se o zprávu serveru, takže klient očekává kód operace
			\item zpráva - podobně jako kód je očekávána klientem
			\item username - jméno hráče volající JOIN\_GAME
		\end{itemize}
\end{itemize}

\subsubsection{Kódy a zprávy}
\begin{itemize}
	\item Pozitivní
		\begin{itemize}
			\item 201 - Successfully joined game
		\end{itemize}
	\item Negativní
		\begin{itemize}
			\item 401 - Server lost your lobby
		\end{itemize}

\end{itemize}
%
%
%
\subsection{Instrukce: OPPONENT\_TURN}
\subsubsection{Popis}
Pošle protivníkovi klientův tah.

\subsubsection{Přesný formát}
\begin{itemize}
	\item ID\_hráče$|$OPPONENT\_TURN$|$kód$|$zpráva$|$parametr\_1$|$parametr\_2$|$...
	\begin{itemize}
			\item kód, zpráva - viz předchozí
			\item parametr\_x - stejně jako u TURN, index v intervalu 0-63
	\end{itemize}

\end{itemize}

\subsubsection{Kódy a zprávy}
\begin{itemize}
	\item Pozitivní
		\begin{itemize}
			\item 201 - Opponent moved
			\item 203 - You won!
			\item 204 - You lost!
		\end{itemize}
\end{itemize}
%
%
%
\subsection{Instrukce: OPPONENT\_DISC}
\subsubsection{Popis}
Informuje hráče o protivníkově odpojení.

\subsubsection{Přesný formát}
\begin{itemize}
	\item ID\_hráče$|$OPPONENT\_DISC$|$kód$|$zpráva
		\begin{itemize}
				\item kód, zpráva - viz instrukce OPPONENT\_JOIN
		\end{itemize}
\end{itemize}

\subsubsection{Kódy a zprávy}
\begin{itemize}
	\item Pozitivní
		\begin{itemize}
			\item 201 - Opponent disconnected
		\end{itemize}
\end{itemize}
%
%
%
\subsection{Instrukce: OPPONENT\_RECO}
\subsubsection{Popis}
Informuje hráče o protivníkově znovu připojení.

\subsubsection{Přesný formát}
\begin{itemize}
	\item ID\_hráče$|$OPPONENT\_RECO$|$kód$|$zpráva
		\begin{itemize}
				\item kód, zpráva - viz instrukce OPPONENT\_JOIN
		\end{itemize}
\end{itemize}

\subsubsection{Kódy a zprávy}
\begin{itemize}
	\item Pozitivní
		\begin{itemize}
			\item 201 - Opponent reconnected
		\end{itemize}
\end{itemize}
%
%
%
\subsection{Instrukce: OPPONENT\_LEFT}
\subsubsection{Popis}
Slouží k informování protivníka o permanentním ukončení protivníkovo spojení.

\subsubsection{Přesný formát}
\begin{itemize}
	\item ID\_hráče$|$OPPONENT\_RECO$|$kód$|$zpráva
		\begin{itemize}
				\item kód, zpráva - viz instrukce OPPONENT\_JOIN
		\end{itemize}
\end{itemize}

\subsubsection{Kódy a zprávy}
\begin{itemize}
	\item Pozitivní
		\begin{itemize}
			\item 201 - Terminating game. Opponent left.
			\end{itemize}
\end{itemize}

%
\chapter{Programátorská dokumentace}
\section{Server}
Je napsán v jazyce C a překládán překladačem gcc pomocí poskytnutého makefile.
\subsection{Struktura serveru}
\begin{figure}[H]
	\centering
	\includegraphics[]{img/server_src_struct.png}
	\caption{Struktura src adresáře serveru}
	\label{img:server_src_struct}
\end{figure}

\subsubsection{Adresář app}
V adresáři app se nacházejí moduly \texttt{automaton, game a player}, ve kterých se nachází struktury 
hry, hráče a automatu. Dále se zde nachází logika hry a obsluhující funkce.
\subsubsection{general\_functions}
Tento modul poskytuje "generickou" strukturu spojového seznamu. Vzhledem k tomu, že data jsou v 
tomto seznamu uložené pomocí void pointeru, umožňuje tento modul pouze přidávat prvky. Práci nad
seznamy už si implementuje každý modul využívající seznam sám. V této práci jsou využity spojové
seznamy pro hráče a hry.
\subsubsection{logger}
Poskytuje funkci zápisu zprávy do souboru.
\subsubsection{main}
Hlavní modul programu obstarávající spouštěcí parametry.
\subsubsection{Adresář network}
Obsahuje moduly \texttt{server, controller, message\_builder}. Server se stará o spuštění
serverového socketu a následných připojení a komunikaci klientů. Komunikaci klientů obsluhuje
controller, který využívá message\_builder pro snadnější vytvoření odpovědi klientovi.

\subsection{Obsluha spojení}
Jak již bylo zmíněno spojení/ připojení a komunikaci obsluhuje modul \texttt{server}. Obsluha více
socketů je realizována použitím \textbf{fd\_set} a funkce \textbf{select}. Program tedy obsluhuje
klienty postupně pouze jedním vláknem.
%
\section{Klient}
Klient je psán v jazyce Java verze 11 s použitím GUI knihovny JavaFX, která do Java 1.8 byla součástí
Java JDK. Ovšem pro verzi Java 11 už je JavaFX samostatnou knihovnou.
\subsection{Struktura klienta}
\begin{figure}[H]
	\centering
	\includegraphics[]{img/klient_src_struct.png}
	\caption{Adresářová struktura src adresáře klienta}
	\label{img:klient_src_struct}
\end{figure}

V adresáři \texttt{java} se nachází podadresáře \texttt{game, graphics, network}. 
V adresáři \texttt{game} se nachází logika hry a funkce obsluhují komunikaci se serverem.
Adresář \texttt{network} obsahuje třídy starající se o navázání spojení a posílání/ přijímání
dat a jejich parsování. Adresář graphics obsahuje grafické ovladače k souborům s příponou .fxml,
které se nacházejí v \texttt{resources/fxml\_res}.
%
\chapter{Uživatelská dokumentace}
\section{Server}
U serveru je předpokládáno, že bude spuštěn na platformě Linux. Server je konzolová aplikace. Překládá
a spouští se z terminálu.
%
\subsection{Překlad}
Server se překládá pomocí 
poskytnutého \texttt{makefile}. V adresáři serveru, kde se nachází \texttt{makefile} stačí
pouze zavolat příkaz \textbf{make} a zdrojové soubory jsou přeloženy překladačem \textbf{gcc}
a vytvoří se spustitelný soubor \textbf{run\_server}.
%
\subsection{Spuštění}
Server se pouští voláním \textbf{run\_server}. Při spuštění je možné použít několik volitelných parametrů.
\begin{itemize}
	\item -h - vypíše nápovědu k parametrům
	\item -a IP - pokusí se vytvořit a spustit socket s adresou IP.
	\item -p PORT - pokusí se vytvořit a spustit socket s portem PORT.
	\item -c NUM - povoluje maximálně NUM počet současných spojení na serveru.
\end{itemize}

Tyto parametry jsou na sobě nezávislé a mohou být použity v libovolném pořadí. Pokud některý parametr
není použit, je použita jeho výchozí hodnota.
Výchozí hodnoty:
\begin{itemize}
	\item IP - 127.0.0.1
	\item PORT - 61116
	\item NUM - 50
\end{itemize}
%
\section{Klient}
Klienta je možné spustit na platformách Linux a Windows. Je psán v Java 11 a překládán pomocí gradle.
\subsection{Překlad}
Aplikace k přeložení používá gradle wrapper, který má závislost na Java. Aplikace je také psána v jazyce 
Java a proto je po uživateli požadováno nainstalovat JDK. Doporučuji nainstalovat 
\textbf{Oracle SE Development Kit 11.0.10}, který je dostupný pro Windows i Linux.

K aplikace je poskytnutý gradle wrapper, který umožňuje téměř automatické přeložení aplikace.
Uživateli stačí pouze v příkazovém řádku zavolat příkaz \textbf{gradlew build} v kořenovém adresáři
klienta (kde se nachází soubor \texttt{gradlew} a \texttt{gradlew.bat}).
%
\subsection{Spuštění}
Po příkazu \textbf{gradlew build} se v kořenovém adresáři klienta vytvoří adresář \texttt{build},
v něm podadresář \texttt{libs} a v tomto podadresáři soubor \texttt{client.jar} 
(výsledná cesta tedy je \texttt{build/libs/client.jar}). Tento soubor je možné spustit z příkazové řádky
pomocí \texttt{java -jar jmeno\_souboru}. A nebo na systému Windows dvojitým poklepáním na tento soubor, za
předpokladu, že systém má nastavenou systémovou proměnou pro spuštění Java 11.
%
\chapter{Závěr}
Obě aplikace splňují hlavní body zadání a jsou stabilní. Není tedy možné je shodit nesprávnou komunikací.
Ovšem stále jsou věci, které se dají vylepšit.

Například na straně serveru, jsou použité spojové seznamy, pro 
uchovávání probíhajících her a připojených hráčů. Tyto struktury, ale nejsou velmi efektivní při hledání jednoho
konkrétního prvku v seznamu, což je akce, kterou server dělá velmi často. Není tedy od věci zvážit, jak by si
výkonnostně vedla například struktura Binary Search Tree.

U klienta by bylo možné vylepšit celkový vzhled grafického rozhraní, který má výchozí vzhled Java FX komponent.

\end{document}
